\documentclass[utf-8]{article}

\title{Proof of 1.3}
\author{Aximaris}
\date{}

\usepackage{amsmath}
\usepackage[a4paper,left=3.18cm,right=3.18cm,top=2.54cm,bottom=2.54cm]{geometry}

\begin{document}
\maketitle

In order to prove $Fib(n)$ is the closest integer to
$\frac{\varphi^n}{\sqrt{5}}$, we need to first prove $Fib(n) =
\frac{\left(\varphi^n-\psi^n\right)}{\sqrt{5}}$, where
$\varphi=\frac{1+\sqrt{5}}{2}$ and $\psi=\frac{1-\sqrt{5}}{2}$.

Here is the definition of $Fib(n)$:
\begin{equation}
  \label{}
  Fib(n) =
  \begin{cases}
    0 & \mbox{if n = 0,} \\
    1 & \mbox{if n = 1,} \\
    Fib(n-1)+Fib(n-2) & \mbox{otherwise.} \\
  \end{cases}
\end{equation}

Consider there is a matrix $Q$ that satisfies $Q
\begin{pmatrix}
  Fib(n-1) \\
  Fib(n)   \\
\end{pmatrix} =
\begin{pmatrix}
  Fib(n)   \\
  Fib(n+1) \\
\end{pmatrix}
$. Use the definition above we can easily find such a $Q$, which is to
solve the equation:
\[
Q \begin{pmatrix}
    Fib(n-1) \\
    Fib(n)   \\
  \end{pmatrix}=
  \begin{pmatrix}
    Fib(n) \\
    Fib(n)+Fib(n-1) \\
  \end{pmatrix}
\]
And thus $Q=
\begin{pmatrix}
  0 & 1 \\
  1 & 1 \\
\end{pmatrix}
$.

Therefore, for any given interger $n$, the formula $Q^n
\begin{pmatrix}
  Fib(0) \\
  Fib(1) \\
\end{pmatrix}=
\begin{pmatrix}
  Fib(n) \\
  Fib(n+1) \\
\end{pmatrix}
$ will tell us $Fib(n)$. Here $
\begin{pmatrix}
  Fib(0) \\
  Fib(1) \\
\end{pmatrix}=
\begin{pmatrix}
  0 \\
  1 \\
\end{pmatrix}
$.

Our idea is to find two eigenvectors of matrix $Q$, and substitute $
\begin{pmatrix}
  0 \\
  1 \\
\end{pmatrix}
$ with the combination of two eigenvectors. Assume they are $v_1
\mbox{ and } v_2$.

Solve the equation:
\begin{equation}
  \label{eq:1}
  \begin{aligned}
    Qv &=\lambda v \\
    (Q-\lambda E)v &= 0 \\
  \end{aligned}
\end{equation}
\begin{equation}
  \label{eq:2}
  \begin{aligned}
    det(Q-\lambda E) &=
    \begin{vmatrix}
      -\lambda & 1 \\
      1 & 1-\lambda \\
    \end{vmatrix}
  \end{aligned}
\end{equation}
\begin{align*}
  \lambda(\lambda -1)-1&=0 \\
  \lambda^2-\lambda-1 &=0 \\
\end{align*}

So we get $\lambda_1=\frac{1+\sqrt{5}}{2} \mbox{ and }
\lambda_2=\frac{1-\sqrt{5}}{2}$, which is $\varphi \mbox{ and
}\psi$. And the corresponding eigenvectors are $v_1=
\begin{pmatrix}
  \frac{1+\sqrt{5}}{2} \\
  \frac{3+\sqrt{5}}{2} \\
\end{pmatrix} \mbox{ and } v_2=
\begin{pmatrix}
  \frac{1-\sqrt{5}}{2} \\
  \frac{3-\sqrt{5}}{2} \\
\end{pmatrix}
$.

Once the eigenvectors have been set, the initial vector of $Fib(n)$, $
\begin{pmatrix}
  0 \\
  1 \\
\end{pmatrix}
$ now can be expressed by the combination of $v_1 \mbox{ and }v_2$.
But consider a better vector $
\begin{pmatrix}
  1 \\
  1 \\
\end{pmatrix}= Q
\begin{pmatrix}
  0 \\
  1 \\
\end{pmatrix}
$
and here we get:
\[
  \begin{pmatrix}
    1 \\
    1 \\
  \end{pmatrix}=\frac{1}{\sqrt{5}}\left(v_1-v_2\right)
\]

And thus, we can tell for any given $n$, the $Fib(n)$ will be:
\begin{align*}
  \begin{pmatrix}
    Fib(n)   \\
    Fib(n+1) \\
  \end{pmatrix} &= Q^n
                  \begin{pmatrix}
                    0 \\
                    1 \\
                  \end{pmatrix} \\
                &= Q^{n-1}
                  \begin{pmatrix}
                    1 \\
                    1 \\
                  \end{pmatrix} \\
                &= \frac{1}{\sqrt{5}}\left(\varphi^{n-1}
                  \begin{pmatrix}
                   \frac{1+\sqrt{5}}{2} \\
                   \frac{3+\sqrt{5}}{2} \\
                  \end{pmatrix} - \psi^{n-1}
                  \begin{pmatrix}
                   \frac{1-\sqrt{5}}{2} \\
                   \frac{3-\sqrt{5}}{2} \\
                  \end{pmatrix}
                  \right)
\end{align*}

Immediately we get $Fib(n)=\frac{\varphi^n-\psi^n}{\sqrt{5}}$.
Notice that $Fib(n)$ will always be an integer.
Therefore, if $\frac{1}{\sqrt{5}}\varphi^n < Fib(n)+ \frac{1}{2}$, then the $Fib(n)$ is
the closest integer to $\varphi^n$.

Notice that
$\frac{1}{\sqrt{5}}\varphi^n=Fib(n)+\frac{1}{\sqrt{5}}\psi^n$, so all
we have to do is to prove:
$$\frac{1}{\sqrt{5}}\psi^n < \frac{1}{2} $$

Clearly,
\[
  \frac{1}{\sqrt{5}} < \frac{1}{2} \mbox{ , and }
  \psi < 1 \mbox{ , which means } \psi^n < 1 \mbox{ too}
\]

Therefore, $\frac{1}{\sqrt{5}}\psi^n < \frac{1}{2}$ is true.

So our final conclusion is that $Fib(n)$ is the closest integer to $\frac{\varphi^n}{\sqrt{5}}$ .
\end{document}